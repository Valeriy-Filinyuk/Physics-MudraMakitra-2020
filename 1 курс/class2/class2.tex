\justifying
\begin{problem}{1}
	Тіло, яке вільно падає, пройшло останні $s = 30$ м за час $\tau = 0.5$ c. З якої висоти $H$ падало тіло?
\end{problem}

\begin{problem}{2}
	Доведіть, що при прямолінійному рівноприскореному русі без початкової швидкості виконується "закон непарних чисел": шляхи, які проходить тіло за послідовні рівні проміжки часу, відносяться як послідовні непарні числа: $s_1: s_2:\dots :s_n = 1:3:\dots :(2n-1)$
\end{problem}

\begin{problem}{3}
	Дві частинки в момент часу $t=0$ вийшли з однієї точки. За графіком залежності (див. рис. \ref{particles}) швидкості від часу визначте координату та час нової зустрічі частинок.
	
	\begin{figure}[h!]
		\centering
		\begin{tikzpicture}
		\begin{axis}[xlabel = {$t$},
		ylabel = {$v, \text{м/с}$}, grid=major]
		\addplot coordinates {
			(0,2) (8,2)
		};
		\addplot coordinates {
			(0,0) (7,2.333)
		};
		\end{axis}
		\end{tikzpicture}
		
		\caption{До задачі \arabic{assigments}.\arabic{problems}}
		\label{particles}
	\end{figure}
\end{problem}

\begin{problem}{4}
	У момент початку спостереження выдстань між двома тілами становила $6.9$ м. Перше тіло рухається зі стану спокою з прискоренням $0.2 ~\dfrac{\text{м}}{\text{c}^2}$. Друге тіло рухається слідом за першим, маючи  початкову швидкість $2 ~\dfrac{\text{м}}{\text{c}}$ і прискорення $0.4 ~\dfrac{\text{м}}{\text{c}^2}$. Написати рівняння $x = x(t)$ у системі відліку, в якій при $t = 0$ координати набувають значень $x_1 = 6.9$ м і $x_2 = 0$. Визначити місце та час зустрічі тіл.
\end{problem}

\begin{problem}{4}
	Від потяга, який рухаєтьс зі сталою швидкістю, відчіпляють останній вагон. Потяг продовжує рухатись з початковою швидкістю. Порівняйте шляхи, які пройшов потяг і вагон до моменту повної зупинки вагона. Прискорення вагона вважайте постійним.
\end{problem}

\begin{problem}{5}
	Тіло падає з висоти $100$ м без початкової швидкості.
	\begin{enumerate}
		\item За який час тіло проходить перший та останній метри свого шляху?
		\item Який шлях тіло проходить за першу секунду свого руху?
		\item за останню?
	\end{enumerate}

\end{problem}
\begin{problem}{7}
	Ліфт починає підійматись з прискоренням $a = 2.2 ~\dfrac{\text{м}}{\text{c}^2}$. Коли його швидкість досягла $v = 2.4 ~\dfrac{\text{м}}{\text{c}}$, зі стелі кабіни ліфта почав падати болт. Чому  дорівнює $t$ падіння болта і переміщення при падінні відносно Землі? Висота кабіни ліфта $H = 2.5$ м
\end{problem}

\begin{problem}{8}
	На похилу площину, площина якої скаладє кут $\alpha$ з горизонтом, поклали масивне тіло. Яке прискорення $a$ (мінімальне) необхідно надати похилій площині в горизонтальному напрямку, щоб вибити її з-під тіла, тобто щоб тіло вільно падало?
\end{problem}
\textbf{Задачі для самостійного розв'язання}

\begin{problem}{4}
	На рис.  відтворено за стробоскопічною фотографією рух кульки по похилому жолобу зі стану спокою. Відомо, що проміжок часк між двома послідовними фотографіями дорівнює $0.2$ с. Поділки дані в дециметрах. Довести, що рух кульки є рівноприскореним. Визначти з яким прискоренням рухалась кулька. Обчислити швидксості кульки в положеннях. зафіксованих на фотографії. \textit{Додатково:} Визначте кут нахилу жолоба
	
	\begin{figure}[h!]
		\centering
		\includegraphics[width=0.5\linewidth]{class2/stroboskop}
		\caption{До задачі \arabic{assigments}.\arabic{problems}}
		\label{fig:stroboskop}
	\end{figure}
	
\end{problem}

\begin{problem}{4}
	Тіло протягом часу $t_0$ ріхається із постійною швидкістю $v_0$. Потім швидкість лінійно зростає з часом так, що в момент часу $2t_0$ вона рівна $2v_0$. Визначте шлях, пройдений тілом за час $t>t_0$

\end{problem}

\begin{problem}{5}
	Два велосипедисти їдуть назустріч один одном Один з них, маючи швидкість $v_{01} = 5.4~\dfrac{\text{км}}{\text{год}}$, спускається з гори, розганяючись з прискоренням $a = 0.2 \dfrac{\text{м}}{\text{c}^2}$; другий, маючи швидкість $v_{02} = 18 \dfrac{\text{км}}{\text{год}}$? піднімається вгору сповільнено з прискоренням $a_2 = 0.2~\dfrac{\text{м}}{\text{c}^2}$. Через який час вони зустрінуться. якщо відстань між ними в початковий момент дорівнювала $s = 195$ м?
\end{problem}

\begin{problem}{6}
	На рис. навдедено графік залежності швидкрсті $v_x(t)$ для тіла, яке рухається вздовж осі $OX$. Побудуйте графіки залежності від часу прискорення $a_x$, переміщення $s_x$, пройденого шляху $l$
	\begin{figure}[h!]
		\centering
		\begin{tikzpicture}
		\begin{axis}[xlabel = {$t$},
		ylabel = {$v_x, \text{м/с}$}, grid=major]
		\addplot coordinates {
			(0,4) (3,-2) (5,-2) (7,0) (9,0)
		};
		
		\end{axis}
		\end{tikzpicture}
		
		\caption{До задачі \arabic{assigments}.\arabic{problems}}
		\label{graph}
	\end{figure}
\end{problem}

\newpage
\begin{problem}{9}
	Потяг пройшов відстань між двома станціями $s = 17$ км з середньою швидкістю $v = 60 \dfrac{\text{км}}{\text{год}}$. При цьому на розгін на початку руху та гальмування перед зупинкою він витратив в сумі $t = 4$ хв. а решту часу рухався з постійною швидкістю $v_1$. Знайдіть $v_1$
\end{problem}
